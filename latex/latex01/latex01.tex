\documentclass[a4j]{jarticle}

%% プリアンブル部 %%

\begin{document}

\title{サルでもわかる\LaTeX 入門}
\author{Christian\thanks{専攻}}
\date{\西暦\today}

\maketitle

\section{\LaTeX とは何か}

\LaTeX は最高級の組版ソフトである. \LaTeX を使えば,数万円のドットプリンタでも数千万円の写植機でも,その能力を最大限に発揮させることができる.



章番号,節番号などを自動的につけることができるし,目次,索引,文献リストも自動的に作れる.
また,脚注も簡単に書ける.

% can comment out using percentage
書体は,和文では明朝とゴシック,欧文では Roman,\textbf{Bold},\textsf{Sans Serif},\textit{Italic},\textsl{Slanted},\textsc{Small Caps},\texttt{Typewriter}などが使える.

また,findのfi,officeのffi,flowerのfl,shuffleのfflのような合字(ligature)の処理,VAX,TOYOTAのような寄せ(kerning)の処理,ハイフン処理(hyphenation)も自動的に行われる.

数式は,なにしろ米国数学会(American Mathematical Society)の標準組版システム\cite{Rate06}になってるくらいであるから,\LaTeX は他のどんなシステムよりも自由度があり,美しい組版が可能である.たとえば

  \[ \int_0^\infty \frac{\sin x}{\sqrt{x}}dx
    = \sqrt{\frac{\pi}{2}} \]

といった数式が簡単に組版できる.
同じ数式でも本文中では $\int_0^\infty$ のように書体が自動的に変わる.
更に,数式中の空白(アキ)も自動的に決めてくれる.
記号 $a=b$ のアキ,足し算 $a+b$ のアキ,符号 $-a$ の後のアキはみな異なる.

\LaTeX の出力は機種に依存しない.
画面,ドットプリンタ,レーザープリンタ,印刷所の写植機でも全く同じ物を出力することができる\cite{HM99}.

\LaTeX のようなソフトを使い慣れてしまうと,もう単純なワープロソフトは使う気になれなくなる(これはちょっと誇大表現だが...).
特に欧文や数式まじりの文章はワープロでは話にならない(これは本当かも).

\section{\LaTeX の作者}

\subsection{Knuthについて}

\LaTeX の作者 Donald E. Knuth は1938年1月10日,アメリカWisconsin州に生まれた.
1960年Case Institute of Technologyを卒業,1963年California Institute of Technologyで博士号(数学)を取得,同大学の教壇にたつ.
1968年からはStanford大学コンピュータ科学科教授を務める\cite{W3TEX}.

\subsection{Knuthの功績}

\begin{itemize}
\item Grace Murray Hopper賞(1971年:ACM)

\item Alan Turing賞(1974年:ACM)

\item Lester R.Ford賞(1975年:MAA)

\item National Medal of Science賞(1979年:USA)

\item McDowell賞(1980年:IEEE)

\item Computer Pioneer賞(1982年:IEEE)
\end{itemize}


\begin{thebibliography}{99}

  
  \bibitem{W3TEX}
  日本語TEX情報, ``http://oku.edu.mie-u.ac.jp/~okumura/texfaq/''.

  \bibitem{Rate06}
  羅手不二子, LATEXとオープンオフィスは寄生虫,KY出版,2006.
  
\end{thebibliography}

\bibliographystyle{plain}
\bibliography{references}

\end{document}