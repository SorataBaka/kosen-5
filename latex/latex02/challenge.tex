\documentclass[a4j, twocolumn]{jarticle}
\usepackage{fancybox}
\usepackage{ascmac}
\usepackage{enumitem}

\begin{document}

\title{釧路の街は栄えるか}
\author{Christian Harjuno\thanks{釧路高専 妄想工学か}}
\date{2022/4/15}

\maketitle

\section{釧路の街危機説}
釧路の町、ヤバいんじゃね?
\begin{itemize}
  \item 基盤産業がない
  \item 人口激減
  \item 高専生の幼稚化
\end{itemize}

\section{釧路の町、安心説}
\begin{enumerate}
  \item 魚まくる(ハズだ!)\cite{GYOGUN}
  \item 温暖化で気候はベストになる(カモ!)
  \item 中国人受けがよい!(といいな!)
\end{enumerate}

\section{釧路の特徴}
\begin{description}
  \item[釧路湿原] 湿原は釧路市の北側に広がる\footnote{面積は約2万 6000ha で、内、中心部の 7863ha がラムサール条約登録温地}。湿原の大半は、北海道川上郡標茶町と阿寒郡鶴居村、釧路郡釧路町に属する。湿原の中を釧路川が¥大きく蛇行しながら流れている\cite{KANKYOU}。
  \item[三大夕日] 釧路湿原を悠々と蛇行する釧路川では水面を赤く染めてキラキラと輝き、その奥の広大な湿原の風景とバックの雌阿寒岳、雄阿寒岳の芝らしいコントラストが見ることができる\cite{KANKYOU}。
  \item[霧の町] 道東の太平洋沿岸では春から夏にかけ海霧がよく発生し、地元の人々はこの霧を「ガス」と呼んでいる。海上から流れてくるこの霧は、釧路の港や街を包み、夜になると四季像のある幣舞橋周辺はロマンチックな雰囲気に変わる\cite{KANKYOU}。 
\end{description}

\section{釧路の超名門大学}
\begin{enumerate}[label=(\arabic*)]
  \item 北海道教育大学釧路校
  \item 釧路公立大学
  \item 釧路短期大学
\end{enumerate}

\section{打開策}
\begin{shadebox}
\begin{enumerate}
  \item 釧路出身のアイドルユニットKSR946を誕生させる。
  \item 巨大原子力発電所を建設し、電気を高値で東京電力に売る。
  \item 無理やり運河とガラス館を作って第二の小樽を目指す。
  \item STAP細胞を捏造する。
\end{enumerate}
\end{shadebox}
\begin{itembox}{高専生の活用}
釧路高専卒業生が将来、大企業を立ち上げ地元釧路に凱旋し、多くの雇用をもたらす。これで、町は活気に溢れ、好景気が爆上がり人口も急激に増加し、ついには新幹線や国際空港が開設され政令指定都市となる。
\end{itembox}

\begin{thebibliography}{99}
  \bibitem{GYOGUN} 魚群喜多蔵、熱いオレの海物事、マリン出版、2013。
  \bibitem{KANKYOU}釧路の観光HP, http://kankou.city.kushiro.hokkaido.jp
\end{thebibliography}

\end{document}