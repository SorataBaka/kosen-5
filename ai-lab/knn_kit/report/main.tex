\documentclass[a4j, twocolumn]{jarticle}
\usepackage{graphicx}
\graphicspath{{./images/}}
\begin{document}

\title{人工知能第1課題}
\author{クリスティアン ハルジュノ 5年情報 21番 234071}
\date{\today}
\maketitle
\section{moon.datの散布図}
\begin{figure}
  \begin{center}
    \includegraphics[width=\linewidth]{moon-scatter.eps}
    \caption{moon.datの散布グラフ図}\label{scatterplot}
  \end{center}
\end{figure}
\section{moon.datを学習パターンファイルとして用い、以下の未知パターンを入力してknn\_mainを実行し、考察せよ\\
(作成した未知パターンファイルおよび実行結果+考察を少なくとも5件含めること)}
\begin{figure}
  \begin{center}
    \includegraphics[width=\linewidth]{moon-boundary.eps}
    \caption{moon.datの決定境界グラフ図}\label{boundarylinegraph}
  \end{center}
\end{figure}
\begin{figure}
  \begin{center}
    \includegraphics[width=\linewidth]{test-data-applied-boundary.eps}
    \caption{moon.datの決定境界グラフ図}\label{boundarylinegraphtestdata}
  \end{center}
\end{figure}
\subsection{moon.datの最初の行の特徴ベクトル →実行結果は0になるはず. なぜそうなるか?}
To understand why the model outputs a value of 0 for the first data point, we need to examine the decision boundary formed after training.
In the second graph, the decision boundary separating the red and blue classes becomes visible. Despite some noise in the data, the model effectively distinguishes the general regions occupied by each class.\\
When we feed the previously seen data back into the trained model, the model recognizes the pattern. As shown in the third graph, the first data point falls within the blue region of the decision boundary. According to the program's definitions, this region corresponds to label 0.\\
Therefore, even if we treat the first data point as unknown, the model will still predict label 0 because it lies in the learned region associated with that label.\\

\subsection{moon.datの最後の行の特徴ベクトル →実行結果は1になるはず. なぜそうなるか?}
As previously explained, the last data in moon.dat is also a known pattern previously used to train the model. When we plot the last data as an unknown data in the graph, it will land on the red area. Based on the program definitions, the red area is considered as the label 1. Therefore, even when we assume that the last data in moon.dat is an unknown pattern, as it has been trained before, the resulting label will always be 1.\\

\subsection{上記1番の特徴ベクトルの数値を僅かに変えた値 →実行結果はどうなるか. なぜそうなるか.}
Changing the vectors but not by much will produce the same result as previously ran. However, if we change it too much so that the new position cross the boundary line, the resulting data will change.\\

\begin{figure}
  \centering
  \includegraphics[width=\linewidth]{tuned-moon-data.eps}
  \caption{変換された特徴ベクトルの境界グラフ上の位置}\label{changed-data-location}
\end{figure}

\begin{table}
  \caption{上記1番特徴ベクトりの数値を変化前後の結果}\label{first-data-modification}
  \centering
  \resizebox{\columnwidth}{!}{
    \begin{tabular}{|c|c|c|c|}
      \hline
       & $\omega_1$ & $\omega_2$ & クラス \\
      \hline
      前&$-0.6475832468935018$&$1.2525427395343274$&$0$\\
      \hline
      後&$0.6475832468935018$&$-1.2525427395343274$&$1$\\
      \hline
    \end{tabular}
  }
\end{table}
In this case, i simply made a negative value positive, and positive value negative. The first feature vector location used to be in the blue area as seen in graph 3. However after changing the values, the vector now resides in the red area as seen in graph 4. The execution result reveals that the class of the first feature vector changed from 0 to 1.\\
\subsection{上記2番の特徴ベクトルの数値を僅かに変えた値 →実行結果はどうなるか. なぜそうなるか.}
\begin{table}[h]
  \caption{上記1番特徴ベクトりの数値を変化前後の結果}\label{second-data-modification}
  \centering
  \resizebox{\columnwidth}{!}{
    \begin{tabular}{|c|c|c|c|}
      \hline
       & $\omega_1$ & $\omega_2$ & クラス \\
      \hline
      前&$0.16534119499451885$&$1.8756075764864084$&$1$\\
      \hline
      後&$-0.16534119499451885$&$-1.8756075764864084$&$0$\\
      \hline
    \end{tabular}
  }
\end{table}
Same as the case before, I switched the value of positive values to negative and negative values to positive. This feature vector which used to reside in the red area based on graph 3, now resides in the blue area as seen in graph 4. Executing this modified data reveals that the class of the second vector changed from 1 to 0.\\

\subsection{重要: その他, 考察のために自由に設定した値. ※最近傍決定則による判断が切り替わると予想される付近の座標を連続で複数与えて探ってみると良い.}
in this experiment, i generated a data that is located vertically in the center and gradually moves from the left to right. The result of plotting this data on the boundary graph can be seen in figure 5.
\begin{figure}
  \centering
  \includegraphics[width=\linewidth]{gradual-data.eps}
  \caption{境界線を越える際の変化を示すための漸進的なデータ}\label{gradual-data-plot}
\end{figure}
\begin{table}[h]
  \centering
  \begin{tabular}{|c|c|c|}
    \hline
    $\omega_1$ & $\omega_2$ & ラベル \\
    \hline
    -2.0 & 0.0 & 0 \\
    -1.8 & 0.0 & 0 \\
    -1.6 & 0.0 & 0 \\
    -1.4 & 0.0 & 0 \\
    -1.2 & 0.0 & 0 \\
    -1.0 & 0.0 & 0 \\
    -0.8 & 0.0 & 0 \\
    -0.6 & 0.0 & 0 \\
    -0.4 & 0.0 & 0 \\
    -0.2 & 0.0 & 0 \\
     0.0 & 0.0 & 0 \\
     0.2 & 0.0 & 1 \\
     0.4 & 0.0 & 1 \\
     0.6 & 0.0 & 1 \\
     0.8 & 0.0 & 1 \\
     1.0 & 0.0 & 1 \\
     1.2 & 0.0 & 1 \\
     1.4 & 0.0 & 1 \\
     1.6 & 0.0 & 1 \\
     1.8 & 0.0 & 1 \\
     2.0 & 0.0 & 1 \\
    \hline
  \end{tabular}
  \caption{境界線を跨ぐときの変化を示す漸進的なデータ}\label{tab:boundary-gradient-data}
\end{table}
As we can see from table 3, as the data gradually moves from the blue area (label 0) to the read area (label 1), the label of the feature vector changes the moment it crosses the line boundary in the middle.\\
When we describe the idea of artificial intelligence, at its core, we are basically trying to generalize the area of a known data so that when we input unknown data, we can predict what its result will be based on the area it lands in.\\

\section{座学の演習問題で自作した2次元データの座標値を定規で読み取り, 学習パターンファイルを作成して決定境界を可視化せよ. (作成した学習パターンファイル, 決定境界の図をレポート)}
\subsection{[ データファイルの形式 ]を参考にして自作学習パターンファイルを作成する. 例: mydata.dat}
\begin{verbatim}
  ファイル名: mydata.dat
  0 2.0 3.4 
  0 3.5 5.5
  0 5.2 4.0
  1 4.7 1.3
  1 6.5 2.3
  1 7.2 0.2
  2 8.0 1.3
  2 7.1 5.3
  2 4.1 6.2
\end{verbatim}
学習パターンを読みやすくするため、表\ref{tab:self-dataset}をご覧ください。
\begin{table}[h]
  \centering
  \begin{tabular}{|c|c|c|}
    \hline
    ラベル & $\omega_1$ & $\omega_2$ \\
    \hline
    0 & 2.0 & 3.4 \\
    0 & 3.5 & 5.5 \\
    0 & 5.2 & 4.0 \\
    1 & 4.7 & 1.3 \\
    1 & 6.5 & 2.3 \\
    1 & 7.2 & 0.2 \\
    2 & 8.0 & 1.3 \\
    2 & 7.1 & 5.3 \\
    2 & 4.1 & 6.2 \\
    \hline
  \end{tabular}
  \caption{作成した既知パターン}\label{tab:self-dataset}
\end{table}
\subsection{決定境界可視化プログラムで自作学習パターンと決定境界を重ねて表示する.}
散布図として作成したパターンを図\ref{self-data-scatter}に参考してください。
\begin{figure}[h]
  \centering
  \includegraphics[width=\linewidth]{self-data-scatter.eps}
  \caption{作成した学習パターンの散布図}\label{self-data-scatter}
\end{figure}
自作学習パターンを境界グラフ作成プログラムで実行すると図\ref{self-data-boundary}のようになります。
\begin{figure}[h]
  \centering
  \includegraphics[width=\linewidth]{self-dat-boundary.eps}
  \caption{作成した学習パターンの境界グラフ}\label{self-data-boundary}
\end{figure}
\section{重要: 課題5で可視化した結果と, 演習問題で手作業で描いた決定境界を比較して考察せよ. 両者で異なる部分や, 自分の理解が間違っていた箇所, 新たな発見等を具体的に書くこと. (レポートに記入)}
Comparing the result of the program to the boundary graph that i drew on my own, i notice the spacing between each vectors are much more clearly defined and spaced evenly. For example, in my own graph, i drew the lines too straight that caused the spacing between boundary and vector to be all over the place. On the other hand, the program was able to come up with more efficient and detailed lines that properly defines the boundary area. Predicting new data using the model that i created by myself would cause unnecessary bias that skews the result of unknown patterns. 
\section{重要: 課題5を参考にしてmoon.datの決定境界も可視化し, 大規模なデータの場合にどの様な決定境界となるか観察し, 考察せよ. (moon.datの決定境界と考察をレポート)}
Decision boundary graph can be seen in figure\ref{boundarylinegraph}. As the amount of data increases, the processing time required to train the model (in this case, determining the decision boundaries) also increases significantly. For a small dataset such as 200 or 300 points, it might take a decent computer a minute or two to fully process the data. However, in some cases, the dataset required to train model might be as big as 100000 points of data. In this case, it can take days or sometimes months to create a fully trained model. 
Running the decision boundary program on moon.dat creates a graph that can be seen in figure\ref{boundarylinegraph}. When training a small dataset, the resulting decision boundary might be straight and blocky as seen in figure\ref{self-data-boundary}. However, when the dataset size is increased, the decision boundaries become much curvier and well-defined. We can look at figure\ref{tensorflow-example} that shows a curvy and well defined boundary graph compared to a straight and blocky one.
\end{document}